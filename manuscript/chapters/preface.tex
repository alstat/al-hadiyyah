\chapter*{Preface}
\label{ch:preface}
\addcontentsline{toc}{chapter}{Preface}
Like many of us, I often find myself questioning the purpose of life. Is it merely about work, money, and fame? Or is there a higher purpose—one that brings tranquility regardless of the circumstances? \txarb{سُبْحَانَ ٱلله}, those reflective moments have guided me to keep striving toward something greater, especially in seeking how I might contribute meaningfully to the Ummah.

It wasn't until I pursued my Master of Science in Statistics that I realized I could apply my skills to uncover patterns in the Qur'\=an and other Islamic texts. My early research led me to the work of Dr. Kais Dukes on the morphological annotation of the Qur'\=an. \txarb{بِإِذْنِ ٱلله}, during my spare time after work, I was able to develop and publish QuranTree.jl, a library built upon Dr. Dukes' contributions. This effort eventually led me to the development of other related tools, such as Yunir.jl (focused on Arabic Natural Language Processing) and Kitab.jl (for accessing texts from Open Islamic Texts Initiative (OpenITI)).

Unfortunately, despite being available for over two years, QuranTree.jl has yet to receive much attention. This is likely due in part to my limited dissemination efforts—such as creating tutorials—and also because I recognized the need for a deeper understanding of Islamic studies. Hence, this work was developed to demonstrate what is possible with the library.

\txarb{ٱلْـحَـمْـدُ للهِ}, I feel truly blessed that, despite the challenges, I am now completing this paper and degree in the sacred lands of Islam—as part of my Hajj, thanks to \textit{Allah} \txarb{\fontspec{Scheherazade New} ﷾} for His call. This journey, \txarb{ٱلْـحَـمْـدُ للهِ}, has been both transformative and inspiring.

Thus, alongside dedicating this work to my beloved father, I would also like to dedicate it to Dr. Kais Dukes and to the  Ummah, whom I hope to inspire to explore the profound beauty of the words of \textit{Allah} \txarb{\fontspec{Scheherazade New} ﷿}, the Most High.
\begin{center}
    Al Asaad\\
    Mina, Saudi Arabia (Dhu'l-Hijjah 07, 1446)\\
\end{center}